\documentclass{article}
\usepackage[nosumlimits]{amsmath}
\usepackage{amssymb,amsthm,MnSymbol}
\def\MM#1{\boldsymbol{#1}}
\newcommand{\pp}[2]{\frac{\partial #1}{\partial #2}} 
\newcommand{\dede}[2]{\frac{\delta #1}{\delta #2}}
\newcommand{\dd}[2]{\frac{\diff#1}{\diff#2}}
\newcommand{\dt}[1]{\diff\!#1}
\def\MM#1{\boldsymbol{#1}}
\DeclareMathOperator{\diff}{d}
\DeclareMathOperator{\Id}{Id}
\DeclareMathOperator{\DIV}{DIV}
\DeclareMathOperator{\D}{D}
\usepackage{amscd}
\usepackage{natbib}
\bibliographystyle{elsarticle-harv}
\usepackage{helvet}
\usepackage{amsfonts}
\renewcommand{\familydefault}{\sfdefault} %% Only if the base font of the docume
\newcommand{\vecx}[1]{\MM{#1}}
\newtheorem{theorem}{Theorem}
\newtheorem{definition}[theorem]{Definition}
\newtheorem{lemma}[theorem]{Lemma}
\newcommand{\code}[1]{{\ttfamily #1}} 
\usepackage[margin=2cm]{geometry}
\newcommand{\jump}[1]{\left[\!\!\left[ #1 \right]\!\!\right]}

\usepackage{fancybox}
\begin{document}
\title{Notes on isometric mean curvature flow}
\author{CJC et al}
\maketitle

Let $\Omega(t)$ be a surface immersed in $\mathbb{R}^3$.  The classical
mean curvature flow solves
\begin{equation}
\dd{}{t}\MM{X} = \kappa(\MM{X})\MM{\nu}(\MM{X}),
\end{equation}
for each point $\MM{X}\in\Omega(t)$, where $\kappa(\MM{X})$ is the
mean curvature of $\Omega(t)$ at $\MM{X}$, defined by
\begin{equation}
\Delta_s \Id = \kappa \MM{\nu}, \quad \mbox{ on }\Omega(t),
\end{equation}
where $\Delta_s$ is the surface Laplacian, $\Id$ is the identity map
on $\Omega(t)$ and $\MM{\nu}$ is the unit oriented normal to
$\Omega(t)$. When equipped with boundary conditions constraining
the boundary of $\Omega(t)$, solutions of this equation can converge
to steady state zero mean curvature surfaces as $t\to \infty$.

A linear in time, finite element in space discretisation of this
equation is to find $\MM{X}^{n+1}\in V$ (where $V$ is a $C^0$
vector-valued finite element space supported on a triangulation of
$\Omega(t^n)$) such that
\begin{equation}
\label{eq:discrete flow}
\langle \MM{X}^{n+1} - \MM{x}, \MM{\MM{\eta}} \rangle
+\Delta t \langle \nabla_s \MM{X}^{n+1}, \nabla_s \MM{\MM{\eta}}\rangle = 0,
\quad \forall \MM{\MM{\eta}}\in V,
\end{equation}
where $\nabla_s$ defines the surface gradient on $\Omega(t^n)$.  After
solving this system, $\MM{X}^{n+1}$ is used to move
$\Omega(t^n)$ to $\Omega(t^{n+1})$ and the numerical integration
continues. This mixture of implicit solution of $\MM{X}^{n+1}$ with
evaluation of $\nabla_s$ at the time level $t^n$ is useful
because it is implicit enough to be stable whilst keeping the equation
linear.

Now we would like to introduce an isometry constraint in a framework
where $\nabla_s$ is represented as an element of $\mathbb{R}^3$, which
happens to have no component in the direction normal to $\MM{\nu}$.
We write $\MM{X}_0(\MM{x})$ as the map from points on $\Omega(t)$ to
points on $\Omega(0)$.
For the map $\MM{x}\to \MM{X}_0(\MM{x})$ to be an isometry, we
require $\nabla_s\MM{X}_0$ to be orthogonal. When calculating
with vectors represented in $\mathbb{R}^3$, this requires the addition
of components normal to $\Omega(t^n)$, i.e. we have the constraint
\begin{equation}
(\nabla \MM{X}_0)^T\cdot \nabla \MM{X}_0 + \MM{\nu}\otimes\MM{\nu} = I,
\end{equation}
where $(\MM{\nu}\otimes\MM{\nu})_{ij}=\nu_i\nu_j$.

Introducing a 2-component-tensor-valued function space $W$, a weak
form of this equation is
\begin{equation} 
0 = F[\chi;\MM{X}_0] = \int_\Omega \chi : \left(
(\nabla\MM{X})^T\cdot\nabla\MM{X} + \MM{\nu}\otimes\MM{\nu} - I\right)\diff x,
\quad \forall \chi \in W.
\end{equation}
We would like to introduce Lagrange multipliers $\Phi\in W$ to
Equation \eqref{eq:discrete flow} to enforce this constraint. The correct
form of the term is obtained by linearising $F$,
\begin{align}
  F'[\chi,\MM{\nu};\MM{X}_0] & =
  \lim_{\epsilon\to 0}\frac{1}{\epsilon}
  \left(F[\chi,\MM{X}_0+\epsilon \MM{\eta}] - F[\chi,\MM{X}_0]\right), \\
  & = 2\int_\Omega \chi:\nabla\MM{X}^T_0\cdot \nabla\MM{\eta}\diff x.
\end{align}
This means that we modify the weak form of the equations to get the
following constrained system,
\begin{align}
\int_\Omega \MM{\nu} \cdot \dot{\MM{X}} \diff x
+\int_\Omega \nabla_s \MM{X}^{n+1} : \nabla_s \MM{\MM{\eta}}\diff x 
- \int_\Omega \Phi : \nabla\MM{X}^T_0\cdot \nabla\MM{\eta}\diff x &= 0,
\quad \forall \MM{\MM{\eta}}\in V, \\
\int_\Omega \chi : \left(
(\nabla\MM{X})^T\cdot\nabla\MM{X} + \MM{\nu}\otimes\MM{\nu} - I\right)\diff x & =
0, 
\quad \forall \chi \in W.
\end{align}
This can form the starting point for a finite element discretiation of
the isometric mean curvature flow. There are some interesting
questions about how to choose finite element spaces for $V$ and $W$ so
that the discretisation is stable (or how to stabilise it with extra
terms if this is not possible).

If everything is smooth enough then we can integrate by parts, and neglect
a surface term (by choosing appropriate boundary conditions for $\MM{X}_0$ or
$\chi$ to obtain the strong form equations
\begin{equation}
  \dot{\MM{X}} = \kappa\MM{\nu} - \nabla\cdot\left(\left(\nabla\MM{X}_0\cdot\Phi\right)^T\right),
  \quad (\nabla \MM{X}_0)^T\cdot \nabla \MM{X}_0 + \MM{\nu}\otimes\MM{\nu} = I.
\end{equation}
I need to spend a bit more time checking whether the constraint term
results in any component in the $\MM{\nu}$ direction, which would mean that
a steady state solution is not a zero curvature surface (i.e., the system
is over-constrained).

\end{document}
